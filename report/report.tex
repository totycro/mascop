\documentclass[,%fontsize=11pt,%
			paper=a4,% 
			%DIV12, % mehr text pro seite als defaultyyp
			DIV12,
			%DIV=calc,%
			%twoside=false,%
			liststotoc,
			bibtotoc,
			draft=false,% final|draft % draft ist platzsparender (kein code, bilder..)
			%titlepage,
			numbers=noendperiod
			]{scrartcl}


\usepackage[utf8]{inputenc}
\usepackage[T1]{fontenc}
\usepackage[english]{babel}

\usepackage{bussproofs}
\usepackage{tikz}
\usetikzlibrary{shapes,arrows,backgrounds,graphs,%
matrix,patterns,arrows,decorations.pathmorphing,decorations.pathreplacing,%
positioning,fit,calc,decorations.text,shadows%
}

\input{/home/totycro/stud/latex-common/latex_header.tex}


%\usepackage{vaucanson-g}
\usepackage{amssymb}
\usepackage{latexsym}

% for color-highlighted code
%\usepackage{color} % for grey comments
%\usepackage{alltt}

%\usepackage[doublespacing]{setspace}
\usepackage[onehalfspacing]{setspace}
%\usepackage[singlespacing]{setspace}
\usepackage{tabularx}
\usepackage{hyperref}
\usepackage{comment}
\usepackage{color}
\usepackage[final]{listings} % sourcecode in document
\usepackage{url}      % for urls
\usepackage{multicol}
\usepackage{float}
\usepackage{caption}
\usepackage{subfigure}
\usepackage{amsmath}
\usepackage{amssymb}

\usepackage{graphicx}

%\usepackage[square]{natbib} % \cite ; square|round etc.
\usepackage[numbers,square]{natbib}
%\usepackage[square, authoryear]{natbib}

%\bibliographystyle{plain}
%\bibliographystyle{alpha}
%\bibliographystyle{alphadin}
%\bibliographystyle{dinat}
%\bibliographystyle{chicago}
%\bibliographystyle{plainnat}

%\bibdata{bib.bib}

\renewcommand*{\partformat}{\partname\ \thepart\ -}
\let\partheadmidvskip\

% smaller url style
\makeatletter
\def\url@leostyle{%
\@ifundefined{selectfont}{\def\UrlFont{\sf}}{\def\UrlFont{\small\ttfamily}}}
\makeatother
\urlstyle{leo}

\newcommand{\myfig}[5] {
 \begin{figure}[tbph]
	 \centering
	 \includegraphics[#3]{#1}
	 \caption[#4]{#5}
	 \label{fig:#2}
 \end{figure}
}

\title{Report on the implementation\\of the Aircraft Landing Problem}
\subtitle{\hfill \\ Modeling and Solving Constrained Optimization Problems}
\author{\hfill \\ Bernhard Mallinger \\ 0707663}
%%\subtitle{}
%\date{13. November 2007}
%\publishers{Betreut durch Univ.-Ass. Dipl.-Ing. Christian Schauer, BSc}

%\usepackage{fancyhdr}
%\setlength{\headrulewidth}{0.0pt}
\pagestyle{plain}

\definecolor{grey}{gray}{.35} % for grey commnts
\lstset{language=Python,%
escapeinside={@}{@},
extendedchars=false,%
%inputencoding=utf8x,%
basicstyle=\ttfamily\small,%
commentstyle=\color{grey},%
%keywordstyle=,% no bold tt in standard font
%captionpos=b,
tabsize=2,
showstringspaces=false,
breaklines=true,
breakindent=0pt,
numbers=left
}

% just for screen-display!
%\usepackage{newcent}

%\newcommand{\ex}[2]{\section*{Exercise #1} \textbf{#2} }
%\newcommand{\ex}[2]{\subsection*{Exercise #1: #2} }

\newcommand{\myt}[1]{\ensuremath{\;\text{ #1 }\;}}
\newcommand{\myts}[1]{\ensuremath{\text{ #1 }}}

\definecolor{grey2}{gray}{.90} 
%\newcommand{\ilc}[1]{\hfil\penalty 100 {\colorbox{grey2}{\texttt{#1}}}}
%\newcommand{\ilc}[1]{{\colorbox{grey2}{\texttt{#1}}}}
\usepackage{soul}
\sethlcolor{grey2}
\newcommand{\ilc}[1]{\hl{\texttt{#1}}}

\usepackage{marginnote}
\reversemarginpar


\begin{document}

\maketitle

\section{The problem}

This exercise deals with implementing the Aircraft Landing Problem, which is specified by the following constraints:

\newcommand{\constr}[1]{\textbf{\textsf{#1}}}
\newcommand{\Ctime}{\constr{TIME}}
\newcommand{\Ccost}{\constr{COST}}
\newcommand{\Crunway}{\constr{RUNWAY}}
\newcommand{\CseqDel}{\constr{SEQUENCE DELAY}}
\newcommand{\CtypeSeq}{\constr{TYPE SEQUENCE}}
\newcommand{\Clandings}{\constr{LANDINGS 30 MINS}}
\begin{description}
	\item[\Ctime] Each aircraft has a minium, preferred and maximum time of landing.
	\item[\Ccost] There are possibly different costs for airplanes being too late or too early.
	\item[\Crunway] Runways can be unavailable for certain periods.
	\item[\CseqDel] Sequences of airplane landings have to keep a certain offset depending on the airplane types.
	\item[\CtypeSeq] There must not be more than 3 landings of the same airplane type in sequence and in any sequence of 5 landings, there must be at most 2 medium sized and 1 large airplane.
	\item[\Clandings] There is a limit to the number of landings that can take place in 30 minutes.

\end{description}

\section{Getting acquainted with the problem, modeling and Gecode}

Due to my lack of major experience in any of the three mentioned fields, I decided to start by relatively freely thinking about and trying different possible ways of modeling.
This phase should not have the concrete demand of yielding useable solutions but to produce experience and generate a feel as to which modeling decisions actually lead to correct results and furthermore which of the ones that are correct are computationally feasible.

%\marginnote{function}
\marginnote{\Ctime}
The first decision was to represent the times when aircrafts land as array of concrete periods \ilc{aircraftTimes} of the length of the number of aircrafts. Each entry corresponds to one aircraft.
This gives a natural way of describing the time frame, in which it is possible for them to land, by specifying the domain for each entry to be this time as given by the instance.

%\marginnote{cost function}
\marginnote{\Ccost}
This representation also allows to model the cost function quite easily.
Since being late and early is punished by different costs, it is necessary to calculate their values separately.
The amount of for instance delay for an aircraft \ilc{i} is given by \ilc{max(0, aircraftTimes[i] - instance.aircraft[i].preferredTime)}.
Using this, an array of delays can be constructed, which is suitable for use with the \ilc{linear}-constraint in combination with the cost vector.
The resulting value summed up with an analoguously constructed value for being too early make up the cost value.

\marginnote{\Clandings}
In order to represent the number of landings within a certain time frame, a different representation has to be found.
As this constraint concerns ranges of periods rather than aircrafts, I employed a set array \ilc{timeAircrafts}, whose entries represent the set of airplanes landing at the respective periods.
It is linked to \ilc{aircraftTimes} by requiring that the index of the airplane \ilc{i} is an element of \ilc{timeAircrafts[ aircraftTimes[ i ] ]}.
Now it's only necessary to sum up the cardinalities of these sets for each time frame of 30 minutes and stating that this sum has to be smaller than the number given by the instance.

Since this simple constraint, if it is tight enough to actually matter, already leads to very bad runtimes for small instances (20-30 planes), I decided to try to vary it in order to allow for better propagation.
In the variation, the union of all sets of airplanes for each 30-minute-frame is calculated in a set variable which whose maximal cardinality is bounded by the limit given by the instance.   
As this change does not change the semantics, the number of solutions and failed nodes stays the same, but depending on the instance, the runtime is sometimes noticably affected with some instances being solved nearly twice as fast whereas others show no significant change.
This suggests that the internal propagation mechanism of Gecode can in some cases deal more efficiently with the second variation.



% branching
Based on these constraints, it is possible to do investigations on the branching parameters. 
For branching, strategies can be selected for which variable to branch on as well as which values of the domain to pick.
It is easily empirically evident that these parameters change the required computation time by multiple orders of magnitude and decide whether a formulation finishes within instant or is computaitonally infeasible.

On instances, that are already tightly constrainted by the \Clandings-constraint, the \ilc{INT_VAR_AFC_MAX}-selector has shown to be exceptionally efficient.


%Since preferred values for landing times of common instances are somewhere between the minima and maxima, choosing values around the extremes of the domain is clearly detrimental and also indeed infeasible even for small instances of around 20 aircrafts.

 



\end{document}
